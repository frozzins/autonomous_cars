%%%%%%%%%%%%%%%%%%%%%%%%%%%%%%%%%%%%%%%%%
% Simple Sectioned Essay Template
% LaTeX Template
%
% This template has been downloaded from:
% http://www.latextemplates.com
%
% Note:
% The \lipsum[#] commands throughout this template generate dummy text
% to fill the template out. These commands should all be removed when 
% writing essay content.
%
%%%%%%%%%%%%%%%%%%%%%%%%%%%%%%%%%%%%%%%%%

%----------------------------------------------------------------------------------------
%	PACKAGES AND OTHER DOCUMENT CONFIGURATIONS
%----------------------------------------------------------------------------------------

\documentclass[12pt]{article} % Default font size is 12pt, it can be changed here
\usepackage{fontspec}
\setmainfont{Arial}

\usepackage{geometry} % Required to change the page size to A4
\geometry{a4paper} % Set the page size to be A4 as opposed to the default US Letter

\usepackage{graphicx} % Required for including pictures

\usepackage{epstopdf}


\usepackage{float} % Allows putting an [H] in \begin{figure} to specify the exact location of the figure
\usepackage{wrapfig} % Allows in-line images such as the example fish picture

\usepackage{lipsum} % Used for inserting dummy 'Lorem ipsum' text into the template

\linespread{1.2} % Line spacing



%\setlength\parindent{0pt} % Uncomment to remove all indentation from paragraphs

%\graphicspath{{Pictures/}} % Specifies the directory where pictures are stored

\DeclareGraphicsExtensions{.pdf,.png,.jpg}

%\graphicspath{{"d:/University of Bristol/Papers/"}{"images/"}}

\usepackage{natbib}

\usepackage{url}

\begin{document}

%----------------------------------------------------------------------------------------
%	TITLE PAGE
%----------------------------------------------------------------------------------------

\begin{titlepage}

\newcommand{\HRule}{\rule{\linewidth}{0.5mm}} % Defines a new command for the horizontal lines, change thickness here

\center % Center everything on the page

\textsc{\LARGE University...}\\[1.5cm] % Name of your university/college
\textsc{\Large Dissertation...}\\[0.5cm] % Major heading such as course name
\textsc{\large Dissertation...}\\[0.5cm] % Minor heading such as course title

\HRule \\[0.4cm]
{ \huge \bfseries Title...}\\[0.4cm] % Title of your document
\HRule \\[1.5cm]

\begin{minipage}{0.4\textwidth}
\begin{flushleft} \large
\emph{Author:}\\
Aleksander \textsc{Wilusz} % Your name
\end{flushleft}
\end{minipage}
~
\begin{minipage}{0.4\textwidth}
\begin{flushright} \large
\emph{Supervisor:} \\
Eddie \textsc{Wilson} % Supervisor's Name
\end{flushright}
\end{minipage}\\[4cm]



{\large \today}\\[3cm] % Date, change the \today to a set date if you want to be precise

%\includegraphics{logouni}\\[1cm] % Include a department/university logo - this will require the graphicx package

\vfill % Fill the rest of the page with whitespace

\end{titlepage}

%----------------------------------------------------------------------------------------
%	TABLE OF CONTENTS
%----------------------------------------------------------------------------------------

\tableofcontents % Include a table of contents

\newpage % Begins the essay on a new page instead of on the same page as the table of contents 

%----------------------------------------------------------------------------------------
%	INTRODUCTION
%----------------------------------------------------------------------------------------


\section{Introduction} % Major section
\paragraph{}
Autonomous cars are currently a topic of great interest. Around the world the biggest private companies or government’s initiatives are developing self-driving vehicles (uber, google, britol car citations), competing for new emerging market. Each company focuses efforts in different direction. The companies are either trying to develope an all-round car that could perform in both city and on highways or restrict the usage to particular types of roads (tesla). Or anyhing in between. The most significant commercial initiatives include Google driverless car, Tesla, Uber (). Goverment founded initiatives include Venturer and ...


\paragraph{•}
Background of the problem, context of the research, reasons why the study was carried out, significance of the study



Many experts around the world are trying to predict how autonomous cars will influence our lives. One of the questions is how the traffic itself will change. Most of the experts agree that in the next decades we will observe gradual process of increasing the share of autonomous cars on our roads. In that time human-driven cars and self driving cars will have to successfully interact with each other. According to predictions the fully automated traffic will become reality only around year 2060 or year 2040 in more optimistic predictions and first commercial autonomous cars are already appearing on the roads (singapore, uber). Although there are numerous studies on many aspect of autonomous driving as well as on interactions between regular cars in all-human-traffic there is little research on interactions between these two types of vehicles and all it’s consequences.

\paragraph{•}
A statement of the problem to be addressed


The focus of this research is to investigate how autonomous cars will influence traffic itself and how human and autonomous cars will interact with each other.




\paragraph{•}
clear and succinct statement of research questions, aims and objectives.


The main aim of the research was trying to predict what will happen when autonomous cars are introduced to the traffic. Results obtained were analysed from the point of view to traffic parameters such as velocities, densities, congestion and from the point of view how individual drivers reacted differently when autonomous vehicle was encountered.

The main objective of the project was to create a an on-line traffic simulation that would allow to connect multiple people together at the same time. People were asked to drive a car 


The original scope of the project included also in-silico implementation that would be using remotely controlled "slot-cars". It was believed that that physical model would have features that could not be accounted for or predicted in computer simulation. It was predicted to account for 50\% of implementation effort

In the final version of the project the physical model was not implemented. 


 How scope
changed. Development calendar


\section{Literature review}


\section{Research Methodology}

Justify the structure of the project. Why the experiment was a key part. Why this was
the best option rather than for example use data from some database?...hmmm

\subsection{Experiment design and implementation}


Justify all major design decisions. Plenty of them!
All the actions undertaken to ensure most meaningful results

\subsubsection{some other subsections}

\subsubsection{Data collection}

How the experiment was eventually conducted

\subsection{Software development}

Justify all major design decisions

\subsubsection{Simulation master design}

\subsubsection{Client's interface design}

General description Simplifications and yet still accounting for most important parts of
the car model

\subsubsection{Car control}

\subsection{Communication between machines}

From one point of view this a tightly coupled with software development but the way
communication was established doesn’t matter from the point of view of software structure.
Simply speaking the comms should only meet some requirements derived from
the main piece of software and the details of implementation doesn’t matter. This was
a significant part of the job and its an achievement on its own.

\subsection{Autonomous car model}

Say why this is important from the point of view of results obtained. Again justify all
design decisions





\begin{figure}[H] % Example image
%\center{\includegraphics[width=1\linewidth]{older_people_uk.eps}}
\caption{Percentage of older people in the UK 1985, 2010, 2035 \citep*{UK_aging}}
\label{fig:senior_65}
\end{figure}

\begin{figure}[H]
  \begin{center}
%    \includegraphics[width=1.0\linewidth]{europe_predictions}
  \end{center}
 \caption{Percentage of persons aged 65 and over EU-27, 1985, 2010, 2035 \citep*{UK_aging}}
\end{figure} 





%\includegraphics*[100,100][300,300]{mypicture}

%\begin{figure}[H] % Example image
%\center{\includegraphics[width=0.5\linewidth]{placeholder}}
%\caption{Example image.}
%\label{fig:speciation}
%\end{figure}




%------------------------------------------------



%\begin{description} % Numbered list example

%\item[First] \hfill \\


%\item[Second] \hfill \\


%\item[Third] \hfill \\


%\end{description} 

%t

%----------------------------------------------------------------------------------------
%	CONCLUSION
%----------------------------------------------------------------------------------------



%----------------------------------------------------------------------------------------
%	BIBLIOGRAPHY
%----------------------------------------------------------------------------------------

\bibliographystyle{unsrt} % vancouver   unsrt   agsm  unsrt
\bibliography{citation_database_new}

%\begin{thebibliography}{99} % Bibliography - this is intentionally simple in this template

%\bibitem[Figueredo and Wolf, 2009]{Figueredo:2009dg}
%Figueredo, A.~J. and Wolf, P. S.~A. (2009).
%\newblock Assortative pairing and life history strategy - a cross-cultural
%  study.
%\newblock {\em Human Nature}, 20:317--330.
 
%\end{thebibliography}





%----------------------------------------------------------------------------------------

\end{document}